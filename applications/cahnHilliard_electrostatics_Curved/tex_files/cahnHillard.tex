\documentclass[10pt]{article}
\usepackage{amsmath}
\usepackage{bm}
\usepackage{bbm}
\usepackage{mathrsfs}
\usepackage{graphicx}
\usepackage{wrapfig}
\usepackage{subcaption}
\usepackage{epsfig}
\usepackage{amsfonts}
\usepackage{amssymb}
\usepackage{amsmath}
\usepackage{wrapfig}
\usepackage{graphicx}
\usepackage{psfrag}
\newcommand{\sun}{\ensuremath{\odot}} % sun symbol is \sun
\let\vaccent=\v % rename builtin command \v{} to \vaccent{}
\renewcommand{\v}[1]{\ensuremath{\mathbf{#1}}} % for vectors
\newcommand{\gv}[1]{\ensuremath{\mbox{\boldmath$ #1 $}}} 
\newcommand{\grad}[1]{\gv{\nabla} #1}
\renewcommand{\baselinestretch}{1.2}
\jot 5mm
\graphicspath{{./figures/}}
%text dimensions
\textwidth 6.5 in
\oddsidemargin .2 in
\topmargin -0.2 in
\textheight 8.5 in
\headheight 0.2in
\overfullrule = 0pt
\pagestyle{plain}
\def\newpar{\par\vskip 0.5cm}
\begin{document}
%
%----------------------------------------------------------------------
%        Define symbols
%----------------------------------------------------------------------
%
\def\iso{\mathbbm{1}}
\def\half{{\textstyle{1 \over 2}}}
\def\third{{\textstyle{1 \over 3}}}
\def\fourth{{\textstyle{{1 \over 4}}}}
\def\twothird{{\textstyle {{2 \over 3}}}}
\def\ndim{{n_{\rm dim}}}
\def\nint{n_{\rm int}}
\def\lint{l_{\rm int}}
\def\nel{n_{\rm el}}
\def\nf{n_{\rm f}}
\def\DIV {\hbox{\af div}}
\def\GRAD{\hbox{\af Grad}}
\def\sym{\mathop{\rm sym}\nolimits}
\def\tr{\mathop{\rm tr}\nolimits}
\def\dev{\mathop{\rm dev}\nolimits}
\def\Dev{\mathop{\rm Dev}\nolimits}
\def\DEV{\mathop {\rm DEV}\nolimits}
\def\bfb {{\bi b}}
\def\Bnabla{\nabla}
\def\bG{{\bi G}}
\def\jmpdelu{{\lbrack\!\lbrack \Delta u\rbrack\!\rbrack}}
\def\jmpudot{{\lbrack\!\lbrack\dot u\rbrack\!\rbrack}}
\def\jmpu{{\lbrack\!\lbrack u\rbrack\!\rbrack}}
\def\jmphi{{\lbrack\!\lbrack\varphi\rbrack\!\rbrack}}
\def\ljmp{{\lbrack\!\lbrack}}
\def\rjmp{{\rbrack\!\rbrack}}
\def\sign{{\rm sign}}
\def\nn{{n+1}}
\def\na{{n+\vartheta}}
\def\nna{{n+(1-\vartheta)}}
\def\nt{{n+{1\over 2}}}
\def\nb{{n+\beta}}
\def\nbb{{n+(1-\beta)}}
%---------------------------------------------------------
%               Bold Face Math Characters:
%               All In Format: \B***** .
%---------------------------------------------------------
\def\bOne{\mbox{\boldmath$1$}}
\def\BGamma{\mbox{\boldmath$\Gamma$}}
\def\BDelta{\mbox{\boldmath$\Delta$}}
\def\BTheta{\mbox{\boldmath$\Theta$}}
\def\BLambda{\mbox{\boldmath$\Lambda$}}
\def\BXi{\mbox{\boldmath$\Xi$}}
\def\BPi{\mbox{\boldmath$\Pi$}}
\def\BSigma{\mbox{\boldmath$\Sigma$}}
\def\BUpsilon{\mbox{\boldmath$\Upsilon$}}
\def\BPhi{\mbox{\boldmath$\Phi$}}
\def\BPsi{\mbox{\boldmath$\Psi$}}
\def\BOmega{\mbox{\boldmath$\Omega$}}
\def\Balpha{\mbox{\boldmath$\alpha$}}
\def\Bbeta{\mbox{\boldmath$\beta$}}
\def\Bgamma{\mbox{\boldmath$\gamma$}}
\def\Bdelta{\mbox{\boldmath$\delta$}}
\def\Bepsilon{\mbox{\boldmath$\epsilon$}}
\def\Bzeta{\mbox{\boldmath$\zeta$}}
\def\Beta{\mbox{\boldmath$\eta$}}
\def\Btheta{\mbox{\boldmath$\theta$}}
\def\Biota{\mbox{\boldmath$\iota$}}
\def\Bkappa{\mbox{\boldmath$\kappa$}}
\def\Blambda{\mbox{\boldmath$\lambda$}}
\def\Bmu{\mbox{\boldmath$\mu$}}
\def\Bnu{\mbox{\boldmath$\nu$}}
\def\Bxi{\mbox{\boldmath$\xi$}}
\def\Bpi{\mbox{\boldmath$\pi$}}
\def\Brho{\mbox{\boldmath$\rho$}}
\def\Bsigma{\mbox{\boldmath$\sigma$}}
\def\Btau{\mbox{\boldmath$\tau$}}
\def\Bupsilon{\mbox{\boldmath$\upsilon$}}
\def\Bphi{\mbox{\boldmath$\phi$}}
\def\Bchi{\mbox{\boldmath$\chi$}}
\def\Bpsi{\mbox{\boldmath$\psi$}}
\def\Bomega{\mbox{\boldmath$\omega$}}
\def\Bvarepsilon{\mbox{\boldmath$\varepsilon$}}
\def\Bvartheta{\mbox{\boldmath$\vartheta$}}
\def\Bvarpi{\mbox{\boldmath$\varpi$}}
\def\Bvarrho{\mbox{\boldmath$\varrho$}}
\def\Bvarsigma{\mbox{\boldmath$\varsigma$}}
\def\Bvarphi{\mbox{\boldmath$\varphi$}}
\def\bone{\mathbf{1}}
\def\bzero{\mathbf{0}}
%---------------------------------------------------------
%               Bold Face Math Italic:
%               All In Format: \b* .
%---------------------------------------------------------
\def\bA{\mbox{\boldmath$ A$}}
\def\bB{\mbox{\boldmath$ B$}}
\def\bC{\mbox{\boldmath$ C$}}
\def\bD{\mbox{\boldmath$ D$}}
\def\bE{\mbox{\boldmath$ E$}}
\def\bF{\mbox{\boldmath$ F$}}
\def\bG{\mbox{\boldmath$ G$}}
\def\bH{\mbox{\boldmath$ H$}}
\def\bI{\mbox{\boldmath$ I$}}
\def\bJ{\mbox{\boldmath$ J$}}
\def\bK{\mbox{\boldmath$ K$}}
\def\bL{\mbox{\boldmath$ L$}}
\def\bM{\mbox{\boldmath$ M$}}
\def\bN{\mbox{\boldmath$ N$}}
\def\bO{\mbox{\boldmath$ O$}}
\def\bP{\mbox{\boldmath$ P$}}
\def\bQ{\mbox{\boldmath$ Q$}}
\def\bR{\mbox{\boldmath$ R$}}
\def\bS{\mbox{\boldmath$ S$}}
\def\bT{\mbox{\boldmath$ T$}}
\def\bU{\mbox{\boldmath$ U$}}
\def\bV{\mbox{\boldmath$ V$}}
\def\bW{\mbox{\boldmath$ W$}}
\def\bX{\mbox{\boldmath$ X$}}
\def\bY{\mbox{\boldmath$ Y$}}
\def\bZ{\mbox{\boldmath$ Z$}}
\def\ba{\mbox{\boldmath$ a$}}
\def\bb{\mbox{\boldmath$ b$}}
\def\bc{\mbox{\boldmath$ c$}}
\def\bd{\mbox{\boldmath$ d$}}
\def\be{\mbox{\boldmath$ e$}}
\def\bff{\mbox{\boldmath$ f$}}
\def\bg{\mbox{\boldmath$ g$}}
\def\bh{\mbox{\boldmath$ h$}}
\def\bi{\mbox{\boldmath$ i$}}
\def\bj{\mbox{\boldmath$ j$}}
\def\bk{\mbox{\boldmath$ k$}}
\def\bl{\mbox{\boldmath$ l$}}
\def\bm{\mbox{\boldmath$ m$}}
\def\bn{\mbox{\boldmath$ n$}}
\def\bo{\mbox{\boldmath$ o$}}
\def\bp{\mbox{\boldmath$ p$}}
\def\bq{\mbox{\boldmath$ q$}}
\def\br{\mbox{\boldmath$ r$}}
\def\bs{\mbox{\boldmath$ s$}}
\def\bt{\mbox{\boldmath$ t$}}
\def\bu{\mbox{\boldmath$ u$}}
\def\bv{\mbox{\boldmath$ v$}}
\def\bw{\mbox{\boldmath$ w$}}
\def\bx{\mbox{\boldmath$ x$}}
\def\by{\mbox{\boldmath$ y$}}
\def\bz{\mbox{\boldmath$ z$}}
%*********************************
%Start main paper
%*********************************
\centerline{\Large{\bf PRISMS PhaseField}}
\smallskip
\centerline{\Large{\bf Cahn-Hilliard Dynamics (Mixed-Formulation)}}
\bigskip
Consider a free energy expression of the form:
\begin{equation}
  \Pi(c, \grad  c) = \int_{\Omega}    f( c ) + \frac{\kappa}{2} \grad  c  \cdot \grad  c    ~dV 
\end{equation}
where $c$ is the composition, and $\kappa$ is the gradient length scale parameter.
	
\section{Variational treatment}
Considering variations on the primal field $c$ of the from $c+\epsilon w$, we have
\begin{align}
\delta \Pi &=  \left. \frac{d}{d\epsilon} \int_{\Omega}  f(c+\epsilon w) +  \frac{\kappa}{2} \grad  (c+\epsilon w)  \cdot  ~\grad  (c+\epsilon w)   ~dV \right\vert_{\epsilon=0} \\
&=  \int_{\Omega}   w f_{,c} +   \kappa \grad w \grad  c    ~dV \\
&=  \int_{\Omega}   w \left( f_{,c} -  \kappa \Delta c \right)  ~dV  +   \int_{\partial \Omega}   w \kappa \grad c \cdot n   ~dS
\end{align}
Assuming $\kappa \grad c \cdot n = 0$, and using standard variational arguments on the equation $\delta \Pi =0$ we have the expression for chemical potential as
\begin{equation}
  \mu  = f_{,c} -  \kappa \Delta c
\end{equation}

\section{Kinetics}
Now the Parabolic PDE for Cahn-Hilliard dynamics is given by:
\begin{align}
  \frac{\partial c}{\partial t} &= -~\grad \cdot (-M\grad \mu)\\
  &=-M~\grad \cdot (-\grad (f_{,c} -  \kappa \Delta c)) 
\end{align}
where $M$ is the constant mobility. This equation can be split into two equations as follow:
\begin{align}
  \mu &= f_{,c} -  \kappa \Delta c \\
  \frac{\partial c}{\partial t} &= M~\grad \cdot (\grad \mu)
\end{align}
\section{Time discretization}
Considering forward Euler explicit time stepping, we have the time discretized kinetics equation:
\begin{align}
  \mu^{n+1} &= f_{,c}^{n} -  \kappa \Delta c^{n} \\
 c^{n+1} &= c^{n} + \Delta t M~\grad \cdot (\grad \mu^{n})
\end{align}

\section{Weak formulation}
In the weak formulation, considering an arbitrary variation $w$, the above equations can be expressed as residual equations representing a mixed (split) formulation:
\begin{align}
  \int_{\Omega}   w  \mu^{n+1}  ~dV &= \int_{\Omega}  w  f_{,c}^{n} - w \kappa \Delta c^{n}~dV\\
  &=\int_{\Omega}  w  \underbrace{f_{,c}^{n}}_{r_{mu}} + \grad w \cdot \underbrace{\kappa \grad c^{n}}_{r_{mux}} ~dV 
\end{align}
and 
\begin{align}
\int_{\Omega}   w c^{n+1} ~dV&= \int_{\Omega}   w c^{n} + w \Delta t M~\grad \cdot (\grad \mu^{n}) ~dV \\
&= \int_{\Omega}   w \underbrace{ c^{n}}_{r_{c}} + \grad w \underbrace{ (-\Delta t M)~ \cdot (\grad \mu^{n})}_{r_{c x}} ~dV \quad \text{[neglecting boundary flux]} 
\end{align}
\vskip 0.25in
The above values of $r_{mu}$, $r_{mux}$, $r_{c}$ and $r_{cx}$ are used to define the residuals in the following parameters file: \\
\textit{applications/cahnHilliard/parameters.h}


\end{document} 