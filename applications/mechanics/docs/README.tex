\documentclass[11pt]{article}
\usepackage[utf8]{inputenc}
\usepackage{geometry}
\geometry{a4paper}
\usepackage{graphicx} 
\usepackage{booktabs} 
\usepackage{array} 
\usepackage{paralist} 
\usepackage{verbatim} 
\usepackage{subfig} 
\usepackage{fancyhdr} 
\usepackage[nodayofweek]{datetime}
\renewcommand{\dateseparator}{\shortdate}
\usepackage{listings}
\pagestyle{fancy}
\renewcommand{\headrulewidth}{0pt} 
\lhead{}\chead{}\rhead{}
\lfoot{}\cfoot{\thepage}\rfoot{}
\usepackage{sectsty}
\allsectionsfont{\sffamily\mdseries\upshape}

% ToC (table of contents)
\usepackage[nottoc,notlof,notlot]{tocbibind}
\usepackage[titles,subfigure]{tocloft} 
\renewcommand{\cftsecfont}{\rmfamily\mdseries\upshape}
\renewcommand{\cftsecpagefont}{\rmfamily\mdseries\upshape}
\usepackage{listings}
\usepackage{color}
\definecolor{dkgreen}{rgb}{0,0.6,0}
\definecolor{gray}{rgb}{0.5,0.5,0.5}
\definecolor{mauve}{rgb}{0.58,0,0.82}

\lstset{frame=tb,
  language=C++,
  aboveskip=3mm,
  belowskip=3mm,
  showstringspaces=false,
  columns=flexible,
  basicstyle={\small\ttfamily},
  numbers=none,
  numberstyle=\tiny\color{gray},
  keywordstyle=\color{blue},
  commentstyle=\color{dkgreen},
  stringstyle=\color{mauve},
  breaklines=true,
  breakatwhitespace=true
  tabsize=3
}
 
\title{mechanics tutorial}
\date{Updated \today} 
\begin{document}
\maketitle

%%%%%%%%%%%%%%%%%%%%%%%%%%%%%%%%%%%%%%%%%%%%%%%%%%%%%%%%%%%%%%%%%%%%%%%%%%%%

\section{Parameters}
\paragraph{}
This section describes the model parameters that the user can access through the \texttt{parameters.h} file in each application folder.  These can be divided between model-specific parameters and those generic to all models, and there are three categories of model-specific parameters: constant coefficients and tensors, bulk free energy functions, and residuals.  In the context of their model, constant coefficients are self-explanatory.  Residuals are used directly in evolution.  Changing them is not recommended unless the user is comfortable with deal.ii and the PRISMS framework.
\subsection{Generic Parameters}
\begin{itemize}
\item \texttt{problemDIM} \\
	Dimension of the problem (e.g. 1D, 2D, 3D)
\item \texttt{spanX} \\
	Length of system in x-direction
\item \texttt{spanY} \\
	Length of system in y-direction.  Not used if \texttt{problemDIM} $< 2$
\item \texttt{spanZ} \\
	Length of system in z-direction.  Not used if \texttt{problemDIM} $< 3$
\item \texttt{refineFactor} \\
	Defines the refinement of the mesh.  There are $2^{\mathrm{refineFactor}}$ elements in each direction in this implementation, and 						$\left( 2^{\mathrm{refineFactor}} \right)^{\mathrm{problemDIM}}$ elements in total.
\item \texttt{finiteElementDegree} \\
	The order of interpolation of the finite element space.  In this case, the order of the Lagrange elements to be used.
\item \texttt{dt} \\
	The simulation timestep.
\item \texttt{numIncrements} \\
	The number of simulation iterations.  Final time is then \texttt{dt}$\cdot$\texttt{numIncrements}.
\item \texttt{writeOutput} \\
	Whether we are writing any output.  Takes a boolean argument, e.g. \texttt{true}.
\item \texttt{skipOutputSteps} \\
	Output will be written every \texttt{skipOutputSteps} iterations.  If \texttt{writeOutput} \texttt{true}, the initial conditions will always be written.
\end{itemize}

%%%%%%%%%%%%%%%%%%%%%%%%%%%%%%%%%%%%%%%%%%%%%%%%%%%%%%%%%%%%%%%%%%%%%%%%%%%

\subsection{\texttt{mechanics} Parameters}
These parameters focus on building the stiffness tensor $C_{ijkl}$ (\texttt{CijklV}), starting from a Young's Modulus (\texttt{Ev}) and Poisson's ratio (\texttt{nuV}).
\[ \mu = \frac{E}{2(1+\nu)} \]
\[ \lambda = \frac{\nu E}{(1+\nu)(1-2\nu)} \]
\[ C_{ijkl} = \mu (\delta_{ik} \delta_{jl}+\delta_{il} \delta_{jk}) + \lambda \delta_{ij} \delta_{kl} \]
%%%%%%%%%%%%%%%%%%%%%%%%%%%%%%%%%%%%%%%%%%%%%%%%%%%%%%%%%%%%%%%%%%%%%%%%%%%%

\section{Boundary Conditions}
\paragraph{}
Boundary conditions are all boundaries fixed $(u=0)$.  Implementation of alternate boundary conditions is an objective for future releases.

\section{Usage}
\paragraph{}
%\begin{itemize}
%$ cmake CMakeLists.txt \\
%$ make \\
%For serial runs: \\
%$ make run \\
%For parallel runs: \\
%$ mpiexec -np nprocs ./main
%\end{itemize}
\noindent
\begin{lstlisting}[language=bash]
  $ make CMakeLists.txt
  $ make
  For serial runs:
  $ make run
  For parallel runs:
  $ mpiexec -np nprocs ./main
\end{lstlisting}
\end{document}